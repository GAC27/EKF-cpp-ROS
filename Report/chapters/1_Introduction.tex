\section{Introduction}
\label{sec:Introduction}
% The very first letter is a 2 line initial drop letter followed
% by the rest of the first word in caps.
% 
% form to use if the first word consists of a single letter:
% \IEEEPARstart{A}{demo} file is ....
% 
% form to use if you need the single drop letter followed by
% normal text (unknown if ever used by the IEEE):
% \IEEEPARstart{A}{}demo file is ....
% 
% Some journals put the first two words in caps:
% \IEEEPARstart{T}{his demo} file is ....
% 
% Here we have the typical use of a "T" for an initial drop letter
% and "HIS" in caps to complete the first word.
\IEEEPARstart{M}{obile} autonomous systems are fundamentally depended on localization. Their motion and task planing require knowledge about the current robot state. For mobile robots the current posture, which includes position and orientation, is an important part of their state. 

For this reason the report to the group project of the experimental part within the "Autonomous Systems" class at Instituto Superior Técnico in 2016/2017 deals with EKF-based localization with a LRF. The objective for the students is to prove their theoretical knowledge on mobile robotics localization in a practical scenario and to gain first experiences with implementing in a robot operating system (ROS) environment. 

This paper is organized as follows. In section \ref{sec:Methods} ... . Afterwards the ... in section \ref{sec:Implementation}. The ... are then discussed in section \ref{sec:Results}. The paper is concluded in section \ref{sec:Conclusion}.